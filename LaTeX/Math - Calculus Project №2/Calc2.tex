\documentclass[12pt, a4paper]{article}

\usepackage[utf8]{inputenc}
\usepackage[english, russian]{babel}
\usepackage{fancyhdr}
\usepackage{amsmath}
\usepackage{amsthm}
\usepackage{float}
\usepackage{graphicx}
\usepackage{pgfplots}
\usepackage{float}
\usepackage{xcolor}
\pgfplotsset{width=\textwidth, compat=1.13}

\usepgfplotslibrary{external}
\usepgfplotslibrary{fillbetween}
\usetikzlibrary{patterns.meta}


\graphicspath{{./}}
\newcommand{\Mod}[1]{\ \mathrm{mod}\ #1}

\usepackage[a4paper, margin=1.5cm]{geometry}

\usepackage{titlesec}
\titlelabel{\thetitle.\quad}

\pagestyle{plain}

\fancypagestyle{firstpage}{%
  \chead{
  МИНИСТЕРСТВО НАУКИ И ВЫСШЕГО ОБРАЗОВАНИЯ РОССИЙСКОЙ ФЕДЕРАЦИИ 
ФЕДЕРАЛЬНОЕ ГОСУДАРСТВЕННОЕ АВТОНОМНОЕ  
ОБРАЗОВАТЕЛЬНОЕ УЧРЕЖДЕНИЕ ВЫСШЕГО ОБРАЗОВАНИЯ\bigskip

«Национальный исследовательский университет ИТМО»\bigskip

ФАКУЛЬТЕТ ПРОГРАММНОЙ ИНЖЕНЕРИИ И КОМПЬЮТЕРНОЙ ТЕХНИКИ 
}
\fancyfoot[CO]{Санкт-Петербург, 2023}%
}



\definecolor{aqua}{HTML}{003844}
\definecolor{peri}{HTML}{5EB1BF}
\definecolor{royal_blue}{HTML}{0A2463}
\definecolor{periwinkle}{HTML}{D8DCFF}
\definecolor{cerulean}{HTML}{247BA0}
\definecolor{bloodred}{HTML}{690500}
\definecolor{imperial_red}{HTML}{FB3640}
\definecolor{purple}{HTML}{511730}
\definecolor{tangerine}{HTML}{FFA781}

\newtheorem*{task}{Условие}
\newtheorem*{finish}{Заключение}

\tikzexternalize
\begin{document}
\newgeometry{top=1.6cm,bottom=1.6cm, left = 1.2cm, right = 1.2cm}

\topskip0pt
\vspace*{0.25\textheight}
\begin{center}
\textbf{\LARGE РАСЧЁТНО-ГРАФИЧЕСКАЯ РАБОТА}

\LARGE по теме

\LARGE <<Ряды Тейлора и Фурье>>

\LARGE по дисциплине

\LARGE <<Математика>>\bigskip

\LARGE Вариант лит. Д
\end{center}
\vspace*{5cm}
\begin{flushright}
\begin{minipage}{.33\linewidth}
\textit{\textbf{Выполнили:}}\\
\begin{tabular}{l l @{\hspace{8pt}-\hspace{8pt}} l l}
Митя ХХ
\end{tabular}

\textit{\textbf{Преподаватель:}}\\
П. К.
\end{minipage}
\end{flushright}


\thispagestyle{firstpage}
\newpage
\tableofcontents

\restoregeometry
\section{1-е Задание. Ряд Тейлора}
\begin{task}
\begin{enumerate}
\item[]
\item[а)] Некоторую функцию разложили в ряд Маклорена и, придав аргументу x определённое
значение, получили  числовой ряд (\ref{ts:1}). Найдите его сумму.
\begin{equation} \label{ts:1}
 \sum_{n=1}^{\infty}\frac{\left(-1\right)^{n-1}}{2^n * n}
\end{equation}

\item[б)] Найдите первообразную функции \ref{ts:2} в виде ряда, используя стандартные разложения
степенных рядов, а также свойства их сложения и умножения.
\begin{equation} \label{ts:2}
f(x)=\ln\left(\frac{1+x}{1-x}\right)
\end{equation}

\item[в)] Найдите первые k членов разложения в степенной ряд решения дифференциального
уравнения \ref{ts:3} при указанных начальных условиях. Изобразите на графике..
\begin{equation} \label{ts:3}
y'=xy+e^y, y(0)=0, k=4
\end{equation}
\end{enumerate}
\end{task}

\subsection{Решение п. а)}
Для нахождение суммы ряда (\ref{ts:1}) рассмотрим следующий функциональный ряд:
\begin{equation} \label{sol:1}
 \sum_{n=1}^{\infty}-\frac{x^n}{n}
\end{equation},
равный исходному ряду (\ref{ts:1}) при $x=-\frac{1}{2}$:
\begin{equation*}
\begin{aligned}
 \sum_{n=1}^{\infty}-\frac{x^n}{n} &= \left|x=-\frac{1}{2}\right| = \\
  &= \sum_{n=1}^{\infty}(-1)^{n+1}\frac{1}{2^n*n} =\\
  &= \sum_{n=1}^{\infty}(-1)^{n-1}\frac{1}{2^n*n}
 \end{aligned}
\end{equation*}

Докажем, что рассматриваемый ряд (\ref{sol:1}) равномерно сходится в окрестноcти точки $x_0 = -\frac{1}{2}$, например на отрезке $\left[-\frac{3}{4};\frac{3}{4}\right]$.

Так как числовой ряд $\sum_{n=1}^{\infty}\left(\frac{3}{4}\right)^n*\frac{1}{n}$ мажорирует ряд (\ref{sol:1}) $\forall x \in \left[-\frac{3}{4};\frac{3}{4}\right]$:
\begin{equation*}
\begin{aligned}
\left|-x^n\frac{1}{n}\right| &\leq \left|\left(\frac{3}{4}\right)^n*\frac{1}{n}\right| \\
\forall x &\in \left[-\frac{3}{4};\frac{3}{4}\right]
\end{aligned}
\end{equation*},
то по признаку Вейерштрасса ряд (\ref{sol:1}) равномерно сходится на отрезке $\left[-\frac{3}{4};\frac{3}{4}\right]$.

Далее заметим, что:
\begin{equation*}
\sum_{n=1}^{\infty}-\frac{x^n}{n} = \sum_{n=1}^{\infty}\int\limits_{0}^x -t^{n-1} dt
\end{equation*}
При этом, так как ряд (\ref{sol:1}) равномерно сходится отрезке $\left[0;x\right]$ при $x \in \left[0;\frac{3}{4}\right]$ и на отрезке $\left[x;0\right]$ при $x \in \left[-\frac{3}{4};0\right]$ (оба отрезка $\subset \left[-\frac{3}{4};\frac{3}{4}\right]$), то по Теореме об интегрировании функционального ряда:
\begin{equation}\label{sol:2}
\begin{aligned}
 \sum_{n=1}^{\infty}\int\limits_{0}^x -t^{n-1} dt &= \int\limits_{0}^x\sum_{n=1}^{\infty} -t^{n-1} dt\\
  \forall x &\in \left[-\frac{3}{4};\frac{3}{4}\right]
\end{aligned}
\end{equation}
Легко заметить, что в подынтегральном выражении в правой части равенства (\ref{sol:2}) находится сумма бесконечно убывающей геометрической прогрессии, равная по общей формуле:
\begin{equation*}
 \sum_{n=1}^{\infty}a_1 * q^n = \frac{a_1}{1-q}
\end{equation*},
где q - знаменатель прогрессии, $a_1$ - первый член прогрессии. Следовательно, для суммы в подынтегральном выражении:
\begin{equation}\label{sol:3}
\begin{aligned}
a_1 = 1 &, q=t\\
\int\limits_{0}^x\sum_{n=1}^{\infty} -t^{n-1} dt &= \int\limits_{0}^x -\frac{1}{1-t} dt =\\
\left.-ln(1-t)\right|_0^x &= ln(1-x) - ln(1)=\\
&= ln(1-x)\\
  \forall x &\in \left[-\frac{3}{4};\frac{3}{4}\right]
\end{aligned}
\end{equation}
Подставляя в полученное выражение исходный $x=-\frac{1}{2}$, получаем:
\begin{equation*}
ln(1-x)= \left|x=-\frac{1}{2}\right| = ln\left(1+\frac{1}{2}\right)=ln\left(\frac{3}{2}\right)
\end{equation*}
Таким образом, приходим к ответу:
\begin{equation*}
 \sum_{n=1}^{\infty}\frac{\left(-1\right)^{n-1}}{2^n * n} = ln\left(\frac{3}{2}\right)
\end{equation*}

\subsection{Решение п. б)}
Воспользовавшись свойством логарифмом представим исходную функцию (\ref{ts:2}) в виде разности логарифмов (при условии $x \neq 1$):
\begin{equation} \label{ts:2:1}
f(x)=\ln\left(\frac{1+x}{1-x}\right) = \ln\left(1+x\right) - \ln\left(1-x\right), x \neq 1
\end{equation}
При этом, известно стандартное разложение в ряд Маклорена функции $f(x)=ln\left(1+x\right)$:
\begin{equation*}
\begin{aligned}
ln\left(1+x\right) &= \sum_{n=1}^{\infty}\frac{(-1)^{n+1}x^n}{n}\\
\forall x &\in \left(-1; 1\right]
\end{aligned}
\end{equation*}
Чтобы разложить слагаемое $ln\left(1-x\right)$ воспользуемся заменой:
\begin{equation*}
\begin{aligned}
\text{Пусть } t &= -x\\
ln(1-x) = ln(1+t) &= \sum_{n=1}^{\infty}\frac{(-1)^{n+1}t^n}{n}\\
& = \sum_{n=1}^{\infty}\frac{(-1)^{n+1}(-x)^n}{n}\\
& = \sum_{n=1}^{\infty}\frac{(-1)^{2n+1}(x)^n}{n}\\
& = \sum_{n=1}^{\infty}-\frac{(x)^n}{n}\\
\forall t &\in \left(-1; 1\right]\\
\forall x &\in \left[-1; 1\right)
\end{aligned}
\end{equation*}
Таким образом, можем разложить данную функцию (\ref{ts:2:1}) в виде рядов:
\begin{equation*}
\begin{aligned}
\ln\left(1+x\right) - \ln\left(1-x\right) &=  \sum_{n=1}^{\infty}\frac{(-1)^{n+1}x^n}{n} - \sum_{n=1}^{\infty}-\frac{(x)^n}{n}\\
\forall x &\in \left(-1; 1\right)
\end{aligned}
\end{equation*}
Раскроем и почленно рассмотрим полученные ряды:
\begin{equation*}
\begin{aligned}
\sum_{n=1}^{\infty}\frac{(-1)^{n+1}x^n}{n} - \sum_{n=1}^{\infty}-\frac{(x)^n}{n} &= \left(\frac{x}{1} - \frac{x^2}{2} + \frac{x^3}{3} - \frac{x^4}{4} + \ldots\right) - \left(-\frac{x}{1}-\frac{x^2}{2}-\frac{x^3}{3}- \frac{x^4}{4} -\ldots \right)\\
&= \frac{x}{1} - \frac{x^2}{2} + \frac{x^3}{3} - \frac{x^4}{4} + \ldots + \frac{x}{1} + \frac{x^2}{2} + \frac{x^3}{3} + \frac{x^4}{4} \ldots = \\
&= 2 * \frac{x}{1} + 2 * \frac{x^3}{3} + \ldots + 2 * \frac{x^{2n-1}}{2n-1} + \ldots\\
&= \sum_{n=1}^\infty 2 * \frac{x^{2n-1}}{2n-1}\\
&\forall x \in \left(-1; 1\right)
\end{aligned}
\end{equation*}
К такому же выводу можно прийти и более строгими рассуждениями. Вновь рассмотрим полученные ряды и найдём разность $a_n$-х элементов (т.е. запишем разность сумм под одним знаком $\sum$):
\begin{equation*}
\begin{aligned}
\sum_{n=1}^{\infty}\frac{(-1)^{n+1}x^n}{n} - \sum_{n=1}^{\infty}-\frac{(x)^n}{n} &= \sum_{n=1}^{\infty} \frac{(-1)^{n+1}x^n}{n} + \frac{(x)^n}{n}\\
&= \sum_{n=1}^{\infty} \frac{((-1)^{n+1}+1) x^n}{n}
\end{aligned}
\end{equation*}
Заметим, что при $n=2, 4, 6, \ldots$ член ряда обнуляется, а при $n=1, 3, 5, \ldots$ общий член равен $a_n = \frac{2x^n}{n}$. Таким образом, верно следующее:
\begin{equation*}
\begin{aligned}
\sum_{n=1}^{\infty} \frac{((-1)^{n+1}+1) x^n}{n} &= \left|n=2k-1\right| = \sum_{k=1}^{\infty} 2*\frac{x^{2k-1}}{2k-1}
\end{aligned}
\end{equation*}
Таким образом, мы разложили исходную функцию (\ref{ts:2}) в степенной ряд, причём множество аргументов, на котором разложение верно совпадает с множеством определения функции ($D(f(x))=\left(-1; 1\right)$):
\begin{equation*}
\begin{aligned}
f(x)=\ln\left(\frac{1+x}{1-x}\right) &= \sum_{n=1}^{\infty} 2*\frac{x^{2n-1}}{2n-1}\\
&\forall x \in \left(-1; 1\right)
\end{aligned}
\end{equation*}
Найдём первообразную исходной функции:
\begin{equation*}
\begin{aligned}
\int f(x) dx = \int\sum_{n=1}^{\infty} 2*\frac{x^{2n-1}}{2n-1} dx
\end{aligned}
\end{equation*}
Заметим, что разложение исходной функции получено как сумма (разность) 2-х СХ на интервале $\left(-1; 1\right)$ рядов (они СХ, потому что являются рядами разложения функции, а для них необходимо условие сходимости на промежутке, на котором они совпадают с функцией), поэтому оно тоже СХ.

Следовательно, по Теореме о равномерной сходимости степенных рядов полученный ряд рязложения сходится равномерно на $\left[-\rho, \rho\right] \subset \left(-1; 1\right), \forall \rho \in \left[0; 1\right)$.

Значит, по Теореме об интегрировании степенных рядов верно следующее:
\begin{equation} \label{sol:2:3}
\begin{aligned}
\int\sum_{n=1}^{\infty} 2*\frac{x^{2n-1}}{2n-1} dx &= \sum_{n=1}^{\infty}\int 2*\frac{x^{2n-1}}{2n-1} dx\\
 \sum_{n=1}^{\infty} \left(2*\frac{x^{2n}}{2n(2n-1)} + C\right) &= \sum_{n=1}^{\infty}\left( \frac{x^{2n}}{n(2n-1)}\right) + C'
\end{aligned}
\end{equation}
Таким образом, первообразная исходной функции равна:
\begin{equation*}
\begin{aligned}
\int f(x) dx =  \sum_{n=1}^{\infty}\left( \frac{x^{2n}}{n(2n-1)}\right) + C'
\end{aligned}
\end{equation*}
\subsection{Решение п. в)}
Решением дифференциального уравнения является некоторая функция (или семейство функций) $y(x)$, производные которой удовлетворяют уравнению. Мы можем синтезировать функцию с определёнными производными при помощи степенного ряда Тейлора.


Для условия (\ref{ts:3}) будем раскладывать некоторую функцию в ряд Маклорена (т.к.в указанном начальном условие дано значение функции в $x_0=0$ )почленно (по заданию необходимо найти первые 4 члена).

Общий вид разложения функции в ряд Маклорена выглядит так:
\begin{equation*}
\begin{aligned}
f(x) = f(0) + \frac{f^{(1)}(0)}{1!}x + \frac{f^{(2)}(0)}{2!}x^2 + \ldots + \frac{f^{(n)}(0)}{n!}x^n + \ldots
\end{aligned}
\end{equation*}
Следовательно, чтобы найти первые 4 члена разложения решения (функции $y$) в степенной ряд необходимо найти значение первых 3 его производных в точке $x=0$:
\begin{equation*}
\begin{aligned}
y'&=xy+e^y, y(0)=0 \Rightarrow\\
y'(0) &= 0 * y(0) + e^{y(0)} = 0 + 1 = 1\\
\hline\\
y''&= y + x*y' + e^y * y' \Rightarrow\\
y''(0)&= y(0) + 0*y'(0) + e^{y(0)} * y'(0) =\\
&= 0 + 0 + 1 * 1 = 1\\
\hline\\
y'''&= y' + y' + y''*x + e^y * y' * y' + y'' * e^y \Rightarrow\\
y'''(0)&= y'(0) + y'(0) + y''(0)*0 + e^{y(0)} * y'(0) * y'(0) + y''(0) * e^{y(0)} =\\
&= 1 + 1 + 0 + 1*1*1+1*1 = 4\\
\hline
\end{aligned}
\end{equation*}
Таким образом, функцию, являющуюся частным решением исходного дифференциального уравнения можно представить как:
\begin{equation*}
\begin{aligned}
y(x) &= 0 + \frac{1}{1!}x+\frac{1}{2!}x^2 + \frac{4}{3!}x^3 + \ldots\\
&= x + \frac{1}{2}x^2+\frac{4}{6}x^3 + \ldots
\end{aligned}
\end{equation*}
\begin{figure}[H]
\begin{tikzpicture}
\begin{axis}[
	axis lines = left,
	xlabel = \(x\),
	ylabel = {\(y\)},
	ymin=-1,
	xmin=-1,
	xmax=1,
	grid=both,
    grid style={line width=.1pt, draw=gray!10},
    major grid style={line width=.2pt,draw=gray!50},
    minor tick num=5,
	axis x line = bottom,
	axis line style ={line width = .3pt},
	ymax=1,
	xtick distance={0.2},
	ytick distance={0.1},
	legend style={at={(0.03,0.9)},anchor=west}
	]

\addplot[
	line width=2pt,
    domain=-1:1,
    samples=20,
    color=periwinkle,
    restrict y to domain=-20:20,
]
{0};
\addplot[
	line width=2pt,
    domain=-1:1,
    samples=20,
    color=peri,
    restrict y to domain=-20:20,
]
{x};
\addplot[
	line width=2pt,
    domain=-1:1,
    samples=100,
    color=royal_blue,
    restrict y to domain=-20:20,
]
{x+1/2*x^2};
\addplot[
	line width=2pt,
    domain=-1:1,
    samples=100,
    color=purple,
    restrict y to domain=-20:20,
]
{x+1/2*x^2+4/6*x^3};
\legend{$y_1=0$, $y_2=0+x$, $y_3=0 + x + \frac{1}{2}x^2$, $y_4=0 + x + \frac{1}{2}x^2+\frac{4}{6}x^3$}
\end{axis}
\end{tikzpicture}
\caption{Графики сумм первых $k=1\dots4$ членов степенного ряда решения исходного дифференциального уравнения.}
\label{gr:1}
\end{figure}
\section{2-е Задание. Ряд Фурье}
\begin{task}
С помощью разложения в ряд Фурье данной функции в интервале $\left(-\pi;\pi\right)$ найдите сумму числового ряда:
\begin{equation} \label{ts2}
\begin{aligned}
f(x)&=x^2\\
\sum_{n=0}^\infty\left(-1\right)&^{n+1}\frac{1}{n^2}
\end{aligned}
\end{equation}
\end{task}
\subsection{Решение}
Представим исходную функцию (\ref{ts2}) тригонометрическим рядом Фурье в интервале $\left(-\pi;\pi\right)$ (из-за того что функция не переодична - представить её рядом Фурье на всей области определения не удастся).

Рассмотрим тригонометрический ряд Фурье в общем виде:
 \begin{equation*}
\begin{aligned}
f(x)=\frac{a_0}{2}+\sum_{n=1}^{\infty}a_n*\cos\left(nx\right)+b_n*\sin\left(nx\right)
\end{aligned}
\end{equation*},
где:
 \begin{equation*}
\begin{aligned}
a_0 &= \frac{1}{\pi}*\int\limits_{-\pi}^{\pi}f(x)dx\\
a_n &= \frac{1}{\pi}*\int\limits_{-\pi}^{\pi}f(x)\cos\left(nx\right)dx\\
b_n &= \frac{1}{\pi}*\int\limits_{-\pi}^{\pi}f(x)\sin\left(nx\right)dx\\
\end{aligned}
\end{equation*}

Заметим, что исходная функция $f(x)$ - чётная $\left(f(-x)=(-x)^2=x^2=f(x)\right)$, а значит,\textbf{ по свойству ряда Фурье: $b_n=0$}, а $a_n = \frac{2}{\pi}*\int\limits_{0}^{\pi}f(x)\cos\left(nx\right)dx$. Найдём коэффициенты $a_0$ и $a_n$:
 \begin{equation*}
\begin{aligned}
a_0 &= \frac{1}{\pi}*\int\limits_{-\pi}^{\pi}x^2 dx=\left.\frac{1}{\pi}*\frac{x^3}{3}\right|_{-\pi}^{\pi}= \frac{2\pi^2}{3}\\
\hline\\
a_n &= \frac{2}{\pi}*\int\limits_{0}^{\pi}x^2\cos\left(nx\right)dx =\\
&=\left|\text{Выч. по частям}\right|=\frac{2}{\pi}*\left(\left.\frac{x^2}{n}*\sin(nx)\right|_0^{\pi} - \frac{1}{n}\int\limits_{0}^{\pi}2x\sin\left(nx\right)dx\right)\\
&=\left|\text{Выч. по частям}\right| =\\
&= \frac{2}{\pi}*\left(\left.\frac{x^2}{n}*\sin(nx)\right|_0^{\pi} - \frac{1}{n}\left(\left.-\frac{2}{n}\cos(nx)*x\right|_0^{\pi} + \frac{2}{n}\int\limits_{0}^{\pi}\cos(nx)dx\right)\right)\\
&=2*\left(\frac{\pi^2\sin(n\pi)}{\pi n}+\frac{2\pi\cos(n\pi)}{n^2\pi}-\frac{2\sin(n\pi)}{n^3\pi}\right)\\
&=2 * \frac{\left(\pi^2n-2\right)\sin(n\pi)+2\pi*n*\cos(n\pi)}{n^3\pi}=\\
&=\left|\sin(n\pi)=0, \cos(n\pi)=(-1)^n\right|=\\
&=4*\frac{(-1)^n}{n^2}\\
\hline
\end{aligned}
\end{equation*}
Представим исходную функцию тригонометрическим рядом:
 \begin{equation}\label{ser}
\begin{aligned}
f(x) &= \frac{a_0}{2}+\sum_{n=1}^{\infty}a_n*\cos(nx)+0*\sin(nx)\\
&= \frac{\pi^2}{3}+ \sum_{n=1}^{\infty}4*\frac{(-1)^n}{n^2}*\cos(nx)\\
&=\frac{\pi^2}{3}-4*\sum_{n=1}^{\infty}\frac{(-1)^{n-1}}{n^2}*\cos(nx)\\
&=\frac{\pi^2}{3}-4*\sum_{n=1}^{\infty}\frac{(-1)^{n+1}}{n^2}*\cos(nx)\\
&\forall x \in \left(-\pi;\pi\right)
\end{aligned}
\end{equation}
\begin{figure}[H]
\begin{tikzpicture}
\begin{axis}[
	axis lines = left,
	xlabel = \(x\),
	ylabel = {\(f(x)\)},
	ymin=-2,
	xmin=-8,
	xmax=8,
	grid=both,
    grid style={line width=.1pt, draw=gray!10},
    major grid style={line width=.2pt,draw=gray!50},
    minor tick num=5,
	axis x line = bottom,
	axis line style ={line width = .3pt},
	ymax=15,
	xtick distance={2},
	ytick distance={2},
	legend style={at={(0.03,0.9)},anchor=west},
	extra x ticks={ -3.14, 3.14},
	extra x tick labels={$-\pi$, $\pi$},
	extra x tick style ={
	grid = none},
	]


\addplot[
	line width=2pt,
    domain=-8:8,
    samples=200,
    color=peri,
    restrict y to domain=-20:20,
]
{pi^2 * 1/3+4*((-1)^1 * (1/1)^2*cos(deg(1*x))+(-1)^2 * (1/2)^2*cos(deg(2*x))+(-1)^3 * (1/3)^2*cos(deg(3*x))+(-1)^4 * (1/4)^2*cos(deg(4*x))+(-1)^5 * (1/5)^2*cos(deg(5*x))+(-1)^6 * (1/6)^2*cos(deg(6*x))+(-1)^7 * (1/7)^2*cos(deg(7*x))+(-1)^8 * (1/8)^2*cos(deg(8*x))+(-1)^9 * (1/9)^2*cos(deg(9*x)))};
\addplot[
	line width=2pt,
	dotted,
    domain=-8:8,
    samples=100,
    color=purple,
    restrict y to domain=-20:20,
]
{x^2};
\draw [dashed, peri, ultra thick] (-3.14,-2) -- (-3.14,9.86) node[circle, solid, peri, fill, inner sep=2pt, ultra thick ]{};
\draw [dashed, peri, ultra thick] (3.14,-2) -- (3.14,9.86) node[circle, solid, peri, fill, inner sep=2pt, ultra thick ]{};
\legend{ $f(x)=\frac{\pi^2}{3}-4*\sum_{n=1}^{\infty}\frac{(-1)^{n+1}}{n^2}*\cos(nx)$,$f(x)=x^2$}
\end{axis}
\end{tikzpicture}
\caption{Графики исходной функции $f(x)$ и её разложения в тригонометрический ряд Фурье.}
\label{gr:2}
\end{figure}
На рисунке \ref{gr:2} видно, что полученное разложение исходной функции в ряд Фурье совпадает с ней на всём конечном интервале  $\left(-\pi;\pi\right)$, так как она непрерывна на всей области определения (отсутствуют разрывы 1-го порядка). Более того, полученное разложение существует и совпадает с функцией и в точках $x=-\pi, x=\pi$.

Заметим, что при $x=0\in\left(-\pi;\pi\right)$ сумма $\sum_{n=1}^{\infty}\frac{(-1)^{n+1}}{n^2}*\cos(nx)$ из (\ref{ser}) совпадает с искомым рядом (т.к. $cos(0)=1$). Зафиксируем точку $x=0$ и найдём искомый ряд:
 \begin{equation*}
\begin{aligned}
f(x)&= \frac{\pi^2}{3}-4*\sum_{n=1}^{\infty}\frac{(-1)^{n+1}}{n^2}*\cos(nx)\\
0^2 &= \frac{\pi^2}{3}-4*\sum_{n=1}^{\infty}\frac{(-1)^{n+1}}{n^2}*\cos(0)\\
0 &= \frac{\pi^2}{3} -4*\sum_{n=1}^{\infty}\frac{(-1)^{n+1}}{n^2}\\
\sum_{n=1}^{\infty}\frac{(-1)^{n+1}}{n^2} &= \frac{\pi^2}{3*4}\\
\sum_{n=1}^{\infty}\frac{(-1)^{n+1}}{n^2} &= \frac{\pi^2}{12}
\end{aligned}
\end{equation*}
Таким образом, сумма искомого числового ряда (\ref{ts2}) равна:
 \begin{equation*}
\begin{aligned}
\sum_{n=1}^{\infty}\frac{(-1)^{n+1}}{n^2} &= \frac{\pi^2}{12}
\end{aligned}
\end{equation*}


\section{3-е Задание. Приближение ряда Фурье конечной суммой}
\begin{task}
В трёх однотипных опытах радиолюбителей на вход цифро-аналогового преобразователя
(ЦАП) был подан короткий цифровой сигнал формы, изображённой на рисунке:
\begin{figure}[H]
\centering
\begin{tikzpicture}[scale=0.7]
\begin{axis}[
	axis lines = left,
	xlabel = \(t\),
	ylabel = {},
	ymin=-1,
	xmin=0,
	xmax=5,
	grid=both,
    grid style={line width=.1pt, draw=gray!10},
    major grid style={line width=.2pt,draw=gray!50},
    minor tick num=5,
	axis x line = bottom,
	axis line style ={line width = .3pt},
	ymax=3,
	xtick distance={1},
	ytick distance={1},
	]
\addplot[
	line width=2pt,
    domain=0:1,
    samples=10,
    color=royal_blue,
    restrict y to domain=-20:20,
]
{0};
\addplot[
	line width=2pt,
    domain=1:3,
    samples=10,
    color=royal_blue,
    restrict y to domain=-20:20,
]
{1};
\addplot[
	line width=2pt,
    domain=3:4,
    samples=10,
    color=royal_blue,
    restrict y to domain=-20:20,
]
{2};

\end{axis}
\end{tikzpicture}
\caption{Форма цифрового сигнала, поданного на ЦАП.}
\label{gr:3}
\end{figure}
В каждом опыте ЦАП был настроен на:

\begin{enumerate}
\item[] 1-й опыт. Обрезание всех гармоник кроме \textbf{0-й и 1-й}.

\item[] 2-й опыт. Обрезание всех гармоник  \textbf{после 3-й}.

\item[] 3-й опыт. Обрезание всех гармоник кроме \textbf{после 10-й}.

\end{enumerate}
\end{task}
\subsection{Решение}
Так как цифровой сигнал можно рассматривать как кусочно-монотонную функцию, представим его как сумму гармоник, то есть в виде тригонометрического ряда Фурье на интервале $\left[0;4\right]$. При этом, так как входной сигнал не переодический, получаемая функция разложения, совпадающая с сигналом на конечном интервале $\left[0;4\right]$, будет иметь полупериод $l=\frac{4-0}{2}=2$.

Кусочно-заданная функция, описывающя форму цифрового сигнала равна:
 \begin{equation}\label{part}
f(x)=
\begin{cases}
0, x\in\left[0;1\right)\\
1, x\in\left[1;3\right)\\
2, x\in\left[3;4\right]\\
\end{cases}
\end{equation}

Рассмотрим тригонометрический ряд Фурье с полупериодом $l$ в общем виде:
 \begin{equation*}
\begin{aligned}
f(x)=\frac{a_0}{2}+\sum_{n=1}^{\infty}a_n*\cos\left(\frac{n\pi}{l}x\right)+b_n*\sin\left(\frac{n\pi}{l}x\right)
\end{aligned}
\end{equation*},
где:
 \begin{equation*}
\begin{aligned}
a_0 &= \frac{1}{l}*\int\limits_{-l}^{l}f(x)dx\\
a_n &= \frac{1}{l}*\int\limits_{-l}^{l}f(x)\cos\left(\frac{n\pi}{l}x\right)dx\\
b_n &= \frac{1}{l}*\int\limits_{-l}^{\l}f(x)\sin\left(\frac{n\pi}{l}x\right)dx\\
\end{aligned}
\end{equation*}
При этом по свойству ряда Фурье пределы интегрирования при вычислении коэффициентов могут быть изменены, при условия сохранения их разности в полный период:
\begin{equation*}
\begin{aligned}
\frac{1}{l}*\int\limits_{-l}^{l}\ldots dx = \frac{1}{l}*\int\limits_{0}^{2l}\ldots dx
\end{aligned}
\end{equation*}
Найдём необходимые коэффициенты Фурье для разложения функции (\ref{part}):
\begin{equation*}
\begin{aligned}
a_0 &= \frac{1}{2}*\int\limits_{0}^{4}f(x)dx\\
\frac{1}{2}*\left(\int\limits_{0}^{1} 0 dx + \int\limits_{1}^{3}1 dx + \int\limits_{3}^{4}2 dx\right) &= \frac{1}{2}*\left( 3 - 1 +8 - 6\right) = 2\\
\hline\\
a_n &= \frac{1}{2}*\int\limits_{0}^{4}f(x)*\cos\left(\frac{n\pi}{2}x\right)dx=\\
\frac{1}{2}*\int\limits_{0}^{1} 0*\cos\left(\frac{n\pi}{2}x\right) dx &+ \frac{1}{2}*\int\limits_{1}^{3}1 * \cos\left(\frac{n\pi}{2}x\right) dx + \frac{1}{2}*\int\limits_{3}^{4}2*\cos\left(\frac{n\pi}{2}x\right) dx=\\
\frac{1}{2}*\frac{2}{n\pi}*\sin(\frac{n\pi}{2} * 3) - \frac{1}{2}*\frac{2}{n\pi}*\sin(\frac{n\pi}{2} * 1) &+ \frac{1}{2}*\frac{4}{n\pi}*\sin(\frac{n\pi}{2} * 4) - \frac{1}{2}*\frac{4}{n\pi}*\sin(\frac{n\pi}{2} * 3)=\\
&-\left(\frac{\sin(\frac{n\pi*3}{2})+\sin(\frac{n\pi*1}{2})}{n\pi}\right)=\\
&=0\\
\hline\\
b_n &= \frac{1}{2}*\int\limits_{0}^{4}f(x)*\sin\left(\frac{n\pi}{2}x\right)dx=\\
\frac{1}{2}*\int\limits_{0}^{1} 0*\sin\left(\frac{n\pi}{2}x\right) dx &+ \frac{1}{2}*\int\limits_{1}^{3}1 * \sin\left(\frac{n\pi}{2}x\right) dx + \frac{1}{2}*\int\limits_{3}^{4}2*\sin\left(\frac{n\pi}{2}x\right) dx=\\
-\frac{1}{2}*\frac{2}{n\pi}*\cos(\frac{n\pi}{2} * 3) + \frac{1}{2}*\frac{2}{n\pi}*\cos(\frac{n\pi}{2} * 1) &- \frac{1}{2}*\frac{4}{n\pi}*\cos(\frac{n\pi}{2} * 4) + \frac{1}{2}*\frac{4}{n\pi}*\cos(\frac{n\pi}{2} * 3)=\\
\frac{1}{n\pi}*\cos(\frac{n\pi}{2} * 3) &+ \frac{1}{n\pi}*\cos(\frac{n\pi}{2}) -\frac{2}{n\pi}*\cos(2n\pi)=\\
&=\frac{1}{n\pi}*\left(\cos(\frac{n\pi}{2} * 3)+ \cos(\frac{n\pi}{2})-2\right)\\
&=\frac{1}{n\pi}*\left(2*\cos(n\pi)*\cos(\frac{n\pi}{2})-2\right)\\
&=\frac{2}{n\pi}*\left((-1)^n*\cos(\frac{n\pi}{2})-1\right)
\end{aligned}
\end{equation*}
Таким образом, можем разложить исходную функцию (\ref{part}) в тригонометричекий ряд Фурье на интервале $\left[0;4\right]$ (при этом, так как в точках $x=1$ и $x=3$ функция терпит разрыв 1-го порядка, сумма полученного ряда разложения в этих точках будет равна не значению функции, а, согласно теореме Дирихле, среднему арифметическому пределов с обеих сторон от них):
 \begin{equation}\label{fourier}
\begin{aligned}
f(x)=1+\sum_{n=1}^{\infty}\frac{2}{n\pi}\left((-1)^n*\cos(\frac{n\pi}{2})-1\right)*\sin(\frac{n\pi}{2}*x)
\end{aligned}
\end{equation}
Далее найдём формы аналагового сигнала на выходе ЦАП для каждого из опытов.
\subsubsection*{1-й опыт}
\begin{figure}[H]
\centering
\begin{tikzpicture}[scale=0.7]
\begin{axis}[
	axis lines = left,
	xlabel = \(x\),
	ylabel = {\(f(x)\)},
	ymin=-1,
	xmin=0,
	xmax=4,
	grid=both,
    grid style={line width=.1pt, draw=gray!10},
    major grid style={line width=.2pt,draw=gray!50},
    minor tick num=5,
	axis x line = bottom,
	axis line style ={line width = .3pt},
	ymax=3,
	xtick distance={1},
	ytick distance={1},
	]
\addplot[
	line width=2pt,
    domain=0:1,
    samples=10,
    color=royal_blue,
    restrict y to domain=-20:20,
]
{0};
\addplot[
	line width=2pt,
    domain=1:3,
    samples=10,
    color=royal_blue,
    restrict y to domain=-20:20,
]
{1};
\addplot[
	line width=2pt,
    domain=3:4,
    samples=10,
    color=royal_blue,
    restrict y to domain=-20:20,
]
{2};
\addplot[
	line width=2pt,
    domain=0:4,
    samples=100,
    color=purple,
    restrict y to domain=-20:20,
]
{1+
+(2/(1*pi))*((-1)^1 * cos(deg(1*pi/2))-1)*sin(deg((1*pi/2)*x))};
\legend{Цифровой сигнал,,, $f(x)=1+\sum_{n=1}^{1}\frac{2}{n\pi}\left((-1)^n*\cos(\frac{n\pi}{2})-1\right)*\sin(\frac{n\pi}{2}*x)$}
\end{axis}
\end{tikzpicture}
\caption{Форма сигнала в 1-м опыте (обрезаны все гармоники кроме 0-й и 1-й).}
\label{exp1}
\end{figure}
\subsubsection*{2-й опыт}
\begin{figure}[H]
\centering
\begin{tikzpicture}[scale=0.7]
\begin{axis}[
	axis lines = left,
	xlabel = \(x\),
	ylabel = {\(f(x)\)},
	ymin=-1,
	xmin=0,
	xmax=4,
	grid=both,
    grid style={line width=.1pt, draw=gray!10},
    major grid style={line width=.2pt,draw=gray!50},
    minor tick num=5,
	axis x line = bottom,
	axis line style ={line width = .3pt},
	ymax=3,
	xtick distance={1},
	ytick distance={1},
	]
\addplot[
	line width=2pt,
    domain=0:1,
    samples=10,
    color=royal_blue,
    restrict y to domain=-20:20,
]
{0};
\addplot[
	line width=2pt,
    domain=1:3,
    samples=10,
    color=royal_blue,
    restrict y to domain=-20:20,
]
{1};
\addplot[
	line width=2pt,
    domain=3:4,
    samples=10,
    color=royal_blue,
    restrict y to domain=-20:20,
]
{2};
\addplot[
	line width=2pt,
    domain=0:4,
    samples=100,
    color=purple,
    restrict y to domain=-20:20,
]
{1+
+(2/(1*pi))*((-1)^1 * cos(deg(1*pi/2))-1)*sin(deg((1*pi/2)*x))+(2/(2*pi))*((-1)^2 * cos(deg(2*pi/2))-1)*sin(deg((2*pi/2)*x))+(2/(3*pi))*((-1)^3 * cos(deg(3*pi/2))-1)*sin(deg((3*pi/2)*x))};
\legend{Цифровой сигнал,,, $f(x)=1+\sum_{n=1}^{3}\frac{2}{n\pi}\left((-1)^n*\cos(\frac{n\pi}{2})-1\right)*\sin(\frac{n\pi}{2}*x)$}
\end{axis}
\end{tikzpicture}
\caption{Форма сигнала во 2-м опыте (обрезаны все гармоники после 3-й).}
\label{exp2}
\end{figure}
\subsubsection*{3-й опыт}
\begin{figure}[H]
\centering
\begin{tikzpicture}[scale=0.7]
\begin{axis}[
	axis lines = left,
	xlabel = \(x\),
	ylabel = {\(f(x)\)},
	ymin=-1,
	xmin=0,
	xmax=4,
	grid=both,
    grid style={line width=.1pt, draw=gray!10},
    major grid style={line width=.2pt,draw=gray!50},
    minor tick num=5,
	axis x line = bottom,
	axis line style ={line width = .3pt},
	ymax=3,
	xtick distance={1},
	ytick distance={1},
	]
\addplot[
	line width=2pt,
    domain=0:1,
    samples=10,
    color=royal_blue,
    restrict y to domain=-20:20,
]
{0};
\addplot[
	line width=2pt,
    domain=1:3,
    samples=10,
    color=royal_blue,
    restrict y to domain=-20:20,
]
{1};
\addplot[
	line width=2pt,
    domain=3:4,
    samples=10,
    color=royal_blue,
    restrict y to domain=-20:20,
]
{2};
\addplot[
	line width=2pt,
    domain=0:4,
    samples=200,
    color=purple,
    restrict y to domain=-20:20,
]
{1+
+(2/(1*pi))*((-1)^1 * cos(deg(1*pi/2))-1)*sin(deg((1*pi/2)*x))+(2/(2*pi))*((-1)^2 * cos(deg(2*pi/2))-1)*sin(deg((2*pi/2)*x))+(2/(3*pi))*((-1)^3 * cos(deg(3*pi/2))-1)*sin(deg((3*pi/2)*x))+(2/(4*pi))*((-1)^4 * cos(deg(4*pi/2))-1)*sin(deg((4*pi/2)*x))+(2/(5*pi))*((-1)^5 * cos(deg(5*pi/2))-1)*sin(deg((5*pi/2)*x))+(2/(6*pi))*((-1)^6 * cos(deg(6*pi/2))-1)*sin(deg((6*pi/2)*x))+(2/(7*pi))*((-1)^7 * cos(deg(7*pi/2))-1)*sin(deg((7*pi/2)*x))+(2/(8*pi))*((-1)^8 * cos(deg(8*pi/2))-1)*sin(deg((8*pi/2)*x))+(2/(9*pi))*((-1)^9 * cos(deg(9*pi/2))-1)*sin(deg((9*pi/2)*x))+(2/(10*pi))*((-1)^10 * cos(deg(10*pi/2))-1)*sin(deg((10*pi/2)*x))};
\legend{Цифровой сигнал,,, $f(x)=1+\sum_{n=1}^{10}\frac{2}{n\pi}\left((-1)^n*\cos(\frac{n\pi}{2})-1\right)*\sin(\frac{n\pi}{2}*x)$}
\end{axis}
\end{tikzpicture}
\caption{Форма сигнала во 3-м опыте (обрезаны все гармоники после 10-й).}
\label{exp3}
\end{figure}
Проанализируем, насколько хорошо выходной аналоговый сигнал ЦАПа соответствует входному цифровому при помощи графика функций ошибок:
\begin{figure}[H]
\centering
\begin{tikzpicture}[scale=0.7]
\begin{axis}[
	axis lines = left,
	xlabel = \(x\),
	ylabel = {\(\Delta f(x)\)},
	ymin=-3,
	xmin=0,
	xmax=4,
	grid=both,
    grid style={line width=.1pt, draw=gray!10},
    major grid style={line width=.2pt,draw=gray!50},
    minor tick num=5,
	axis x line = bottom,
	axis line style ={line width = .3pt},
	ymax=3,
	xtick distance={1},
	ytick distance={1},
	]
\addplot[
	line width=2pt,
    domain=0:1,
    samples=100,
    color=royal_blue,
    restrict y to domain=-20:20,
]
{0-1-(2/(1*pi))*((-1)^1 * cos(deg(1*pi/2))-1)*sin(deg((1*pi/2)*x))};
\addplot[
	line width=2pt,
    domain=1:3,
    samples=100,
    color=royal_blue,
    restrict y to domain=-20:20,
]
{1-1-(2/(1*pi))*((-1)^1 * cos(deg(1*pi/2))-1)*sin(deg((1*pi/2)*x))};
\addplot[
	line width=2pt,
    domain=3:4,
    samples=100,
    color=royal_blue,
    restrict y to domain=-20:20,
]
{2-1-(2/(1*pi))*((-1)^1 * cos(deg(1*pi/2))-1)*sin(deg((1*pi/2)*x)) };

\addplot[
	line width=2pt,
    domain=0:1,
    samples=100,
    color=purple,
    restrict y to domain=-20:20,
]
{0-1-(2/(1*pi))*((-1)^1 * cos(deg(1*pi/2))-1)*sin(deg((1*pi/2)*x))-(2/(2*pi))*((-1)^2 * cos(deg(2*pi/2))-1)*sin(deg((2*pi/2)*x))-(2/(3*pi))*((-1)^3 * cos(deg(3*pi/2))-1)*sin(deg((3*pi/2)*x))};
\addplot[
	line width=2pt,
    domain=1:3,
    samples=100,
    color=purple,
    restrict y to domain=-20:20,
]
{1-1-(2/(1*pi))*((-1)^1 * cos(deg(1*pi/2))-1)*sin(deg((1*pi/2)*x))-(2/(2*pi))*((-1)^2 * cos(deg(2*pi/2))-1)*sin(deg((2*pi/2)*x))-(2/(3*pi))*((-1)^3 * cos(deg(3*pi/2))-1)*sin(deg((3*pi/2)*x))};
\addplot[
	line width=2pt,
    domain=3:4,
    samples=100,
    color=purple,
    restrict y to domain=-20:20,
]
{2-1-(2/(1*pi))*((-1)^1 * cos(deg(1*pi/2))-1)*sin(deg((1*pi/2)*x))-(2/(2*pi))*((-1)^2 * cos(deg(2*pi/2))-1)*sin(deg((2*pi/2)*x))-(2/(3*pi))*((-1)^3 * cos(deg(3*pi/2))-1)*sin(deg((3*pi/2)*x))};


\addplot[
	line width=2pt,
    domain=0:1,
    samples=100,
    color=tangerine,
    restrict y to domain=-20:20,
]
{0-1-(2/(1*pi))*((-1)^1 * cos(deg(1*pi/2))-1)*sin(deg((1*pi/2)*x))-(2/(2*pi))*((-1)^2 * cos(deg(2*pi/2))-1)*sin(deg((2*pi/2)*x))-(2/(3*pi))*((-1)^3 * cos(deg(3*pi/2))-1)*sin(deg((3*pi/2)*x))-(2/(4*pi))*((-1)^4 * cos(deg(4*pi/2))-1)*sin(deg((4*pi/2)*x))-(2/(5*pi))*((-1)^5 * cos(deg(5*pi/2))-1)*sin(deg((5*pi/2)*x))-(2/(6*pi))*((-1)^6 * cos(deg(6*pi/2))-1)*sin(deg((6*pi/2)*x))-(2/(7*pi))*((-1)^7 * cos(deg(7*pi/2))-1)*sin(deg((7*pi/2)*x))-(2/(8*pi))*((-1)^8 * cos(deg(8*pi/2))-1)*sin(deg((8*pi/2)*x))-(2/(9*pi))*((-1)^9 * cos(deg(9*pi/2))-1)*sin(deg((9*pi/2)*x))-(2/(10*pi))*((-1)^10 * cos(deg(10*pi/2))-1)*sin(deg((10*pi/2)*x))  };
\addplot[
	line width=2pt,
    domain=1:3,
    samples=100,
    color=tangerine,
    restrict y to domain=-20:20,
]
{1-1-(2/(1*pi))*((-1)^1 * cos(deg(1*pi/2))-1)*sin(deg((1*pi/2)*x))-(2/(2*pi))*((-1)^2 * cos(deg(2*pi/2))-1)*sin(deg((2*pi/2)*x))-(2/(3*pi))*((-1)^3 * cos(deg(3*pi/2))-1)*sin(deg((3*pi/2)*x))-(2/(4*pi))*((-1)^4 * cos(deg(4*pi/2))-1)*sin(deg((4*pi/2)*x))-(2/(5*pi))*((-1)^5 * cos(deg(5*pi/2))-1)*sin(deg((5*pi/2)*x))-(2/(6*pi))*((-1)^6 * cos(deg(6*pi/2))-1)*sin(deg((6*pi/2)*x))-(2/(7*pi))*((-1)^7 * cos(deg(7*pi/2))-1)*sin(deg((7*pi/2)*x))-(2/(8*pi))*((-1)^8 * cos(deg(8*pi/2))-1)*sin(deg((8*pi/2)*x))-(2/(9*pi))*((-1)^9 * cos(deg(9*pi/2))-1)*sin(deg((9*pi/2)*x))-(2/(10*pi))*((-1)^10 * cos(deg(10*pi/2))-1)*sin(deg((10*pi/2)*x)) };
\addplot[
	line width=2pt,
    domain=3:4,
    samples=100,
    color=tangerine,
    restrict y to domain=-20:20,
]
{2-1-(2/(1*pi))*((-1)^1 * cos(deg(1*pi/2))-1)*sin(deg((1*pi/2)*x))-(2/(2*pi))*((-1)^2 * cos(deg(2*pi/2))-1)*sin(deg((2*pi/2)*x))-(2/(3*pi))*((-1)^3 * cos(deg(3*pi/2))-1)*sin(deg((3*pi/2)*x))-(2/(4*pi))*((-1)^4 * cos(deg(4*pi/2))-1)*sin(deg((4*pi/2)*x))-(2/(5*pi))*((-1)^5 * cos(deg(5*pi/2))-1)*sin(deg((5*pi/2)*x))-(2/(6*pi))*((-1)^6 * cos(deg(6*pi/2))-1)*sin(deg((6*pi/2)*x))-(2/(7*pi))*((-1)^7 * cos(deg(7*pi/2))-1)*sin(deg((7*pi/2)*x))-(2/(8*pi))*((-1)^8 * cos(deg(8*pi/2))-1)*sin(deg((8*pi/2)*x))-(2/(9*pi))*((-1)^9 * cos(deg(9*pi/2))-1)*sin(deg((9*pi/2)*x))-(2/(10*pi))*((-1)^10 * cos(deg(10*pi/2))-1)*sin(deg((10*pi/2)*x))};
\legend{1-й опыт,,, 2-й опыт,,, 3-й опыт}
\end{axis}
\end{tikzpicture}
\caption{Погрешность приближения цифрового сигнала аналаговым с выхода ЦАП.}
\label{exp5}
\end{figure}


\begin{figure}[H]
\centering
\begin{tikzpicture}[scale=0.7]
\begin{axis}[
	axis lines = left,
	xlabel = \(x\),
	ylabel = {\(\left|\Delta f(x)\right|\)},
	ymin=0,
	xmin=0,
	xmax=4,
	grid=both,
    grid style={line width=.1pt, draw=gray!10},
    major grid style={line width=.2pt,draw=gray!50},
    minor tick num=5,
	axis x line = bottom,
	axis line style ={line width = .3pt},
	ymax=3,
	xtick distance={1},
	ytick distance={1},
	]
\addplot[
	line width=2pt,
    domain=0:1,
    samples=100,
    color=royal_blue,
    restrict y to domain=-20:20,
]
{abs(0-1-(2/(1*pi))*((-1)^1 * cos(deg(1*pi/2))-1)*sin(deg((1*pi/2)*x)))};
\addplot[
	line width=2pt,
    domain=1:3,
    samples=100,
    color=royal_blue,
    restrict y to domain=-20:20,
]
{abs(1-1-(2/(1*pi))*((-1)^1 * cos(deg(1*pi/2))-1)*sin(deg((1*pi/2)*x)))};
\addplot[
	line width=2pt,
    domain=3:4,
    samples=100,
    color=royal_blue,
    restrict y to domain=-20:20,
]
{abs(2-1-(2/(1*pi))*((-1)^1 * cos(deg(1*pi/2))-1)*sin(deg((1*pi/2)*x))) };

\addplot[
	line width=2pt,
    domain=0:1,
    samples=100,
    color=purple,
    restrict y to domain=-20:20,
]
{abs(0-1-(2/(1*pi))*((-1)^1 * cos(deg(1*pi/2))-1)*sin(deg((1*pi/2)*x))-(2/(2*pi))*((-1)^2 * cos(deg(2*pi/2))-1)*sin(deg((2*pi/2)*x))-(2/(3*pi))*((-1)^3 * cos(deg(3*pi/2))-1)*sin(deg((3*pi/2)*x)))};
\addplot[
	line width=2pt,
    domain=1:3,
    samples=100,
    color=purple,
    restrict y to domain=-20:20,
]
{abs(1-1-(2/(1*pi))*((-1)^1 * cos(deg(1*pi/2))-1)*sin(deg((1*pi/2)*x))-(2/(2*pi))*((-1)^2 * cos(deg(2*pi/2))-1)*sin(deg((2*pi/2)*x))-(2/(3*pi))*((-1)^3 * cos(deg(3*pi/2))-1)*sin(deg((3*pi/2)*x)))};
\addplot[
	line width=2pt,
    domain=3:4,
    samples=100,
    color=purple,
    restrict y to domain=-20:20,
]
{abs(2-1-(2/(1*pi))*((-1)^1 * cos(deg(1*pi/2))-1)*sin(deg((1*pi/2)*x))-(2/(2*pi))*((-1)^2 * cos(deg(2*pi/2))-1)*sin(deg((2*pi/2)*x))-(2/(3*pi))*((-1)^3 * cos(deg(3*pi/2))-1)*sin(deg((3*pi/2)*x)))};


\addplot[
	line width=2pt,
    domain=0:1,
    samples=100,
    color=tangerine,
    restrict y to domain=-20:20,
]
{abs(0-1-(2/(1*pi))*((-1)^1 * cos(deg(1*pi/2))-1)*sin(deg((1*pi/2)*x))-(2/(2*pi))*((-1)^2 * cos(deg(2*pi/2))-1)*sin(deg((2*pi/2)*x))-(2/(3*pi))*((-1)^3 * cos(deg(3*pi/2))-1)*sin(deg((3*pi/2)*x))-(2/(4*pi))*((-1)^4 * cos(deg(4*pi/2))-1)*sin(deg((4*pi/2)*x))-(2/(5*pi))*((-1)^5 * cos(deg(5*pi/2))-1)*sin(deg((5*pi/2)*x))-(2/(6*pi))*((-1)^6 * cos(deg(6*pi/2))-1)*sin(deg((6*pi/2)*x))-(2/(7*pi))*((-1)^7 * cos(deg(7*pi/2))-1)*sin(deg((7*pi/2)*x))-(2/(8*pi))*((-1)^8 * cos(deg(8*pi/2))-1)*sin(deg((8*pi/2)*x))-(2/(9*pi))*((-1)^9 * cos(deg(9*pi/2))-1)*sin(deg((9*pi/2)*x))-(2/(10*pi))*((-1)^10 * cos(deg(10*pi/2))-1)*sin(deg((10*pi/2)*x)) ) };
\addplot[
	line width=2pt,
    domain=1:3,
    samples=100,
    color=tangerine,
    restrict y to domain=-20:20,
]
{abs(1-1-(2/(1*pi))*((-1)^1 * cos(deg(1*pi/2))-1)*sin(deg((1*pi/2)*x))-(2/(2*pi))*((-1)^2 * cos(deg(2*pi/2))-1)*sin(deg((2*pi/2)*x))-(2/(3*pi))*((-1)^3 * cos(deg(3*pi/2))-1)*sin(deg((3*pi/2)*x))-(2/(4*pi))*((-1)^4 * cos(deg(4*pi/2))-1)*sin(deg((4*pi/2)*x))-(2/(5*pi))*((-1)^5 * cos(deg(5*pi/2))-1)*sin(deg((5*pi/2)*x))-(2/(6*pi))*((-1)^6 * cos(deg(6*pi/2))-1)*sin(deg((6*pi/2)*x))-(2/(7*pi))*((-1)^7 * cos(deg(7*pi/2))-1)*sin(deg((7*pi/2)*x))-(2/(8*pi))*((-1)^8 * cos(deg(8*pi/2))-1)*sin(deg((8*pi/2)*x))-(2/(9*pi))*((-1)^9 * cos(deg(9*pi/2))-1)*sin(deg((9*pi/2)*x))-(2/(10*pi))*((-1)^10 * cos(deg(10*pi/2))-1)*sin(deg((10*pi/2)*x)) )};
\addplot[
	line width=2pt,
    domain=3:4,
    samples=100,
    color=tangerine,
    restrict y to domain=-20:20,
]
{abs(2-1-(2/(1*pi))*((-1)^1 * cos(deg(1*pi/2))-1)*sin(deg((1*pi/2)*x))-(2/(2*pi))*((-1)^2 * cos(deg(2*pi/2))-1)*sin(deg((2*pi/2)*x))-(2/(3*pi))*((-1)^3 * cos(deg(3*pi/2))-1)*sin(deg((3*pi/2)*x))-(2/(4*pi))*((-1)^4 * cos(deg(4*pi/2))-1)*sin(deg((4*pi/2)*x))-(2/(5*pi))*((-1)^5 * cos(deg(5*pi/2))-1)*sin(deg((5*pi/2)*x))-(2/(6*pi))*((-1)^6 * cos(deg(6*pi/2))-1)*sin(deg((6*pi/2)*x))-(2/(7*pi))*((-1)^7 * cos(deg(7*pi/2))-1)*sin(deg((7*pi/2)*x))-(2/(8*pi))*((-1)^8 * cos(deg(8*pi/2))-1)*sin(deg((8*pi/2)*x))-(2/(9*pi))*((-1)^9 * cos(deg(9*pi/2))-1)*sin(deg((9*pi/2)*x))-(2/(10*pi))*((-1)^10 * cos(deg(10*pi/2))-1)*sin(deg((10*pi/2)*x)))};
\legend{1-й опыт,,, 2-й опыт,,, 3-й опыт}
\end{axis}
\end{tikzpicture}
\caption{Модуль погрешности приближения цифрового сигнала аналаговым с выхода ЦАП.}
\label{diff}
\end{figure}
При рассмотрении рис. \ref{diff} легко заметить, что при увеличении задействованных в выходном сигнале гармоник, отклонение этого сигнала от входного уменьшается. При этом, в местах резких скачков входного сигнала ($x=1,3$) погрешность у сигналов из всех опытов довольно высока, а на концах сигнала, в точках ($x=0,4$) - она достигает максимальных значений (это связано с тем, что входной непериодичный цифровой сигнал представляется выходным периодичным, а разность концов входного сигнала ($2-0=2$) - велика).
\begin{finish}
\begin{enumerate}
\item[]

\item Точность представления кусочно-монотонной функции тригонометрическим многочленом Фурье прямо зависит от количества взятых гармоник.

\item Погрешность представления многочленом Фурье максимальна в точках разыва исходной функции и на концах промежутка её рассмотрения.
\end{enumerate}

\end{finish}
\section{Выводы}
При выполнении расчётно-графической работы мы нашли значение разложенной в ряд Маклорена функции при помощи перехода к функциональному ряду и интегрирования, разложили функцию при помощи стандартных разложений в ряд Маклорена и нашли решение дифференциального уравнения  как разложенную в ряд Маклорена функцию. Мы также воспользовались разложением функции в ряд Фурье для поиска суммы другого ряда и преобразования цифрвого сигнала, заданного кусочно-монотонной функцией, в аналаговый.
\newpage
\section{Оценочный лист}
\begin{center}
\large
\begin{tabular}{|l|l|l|l|r|}
\hline
Хороших & Дмитрий & P3117 & 1.5 & 100\%\\
\hline
Рамеев & Тимур & P3118 & 1.5  & 100\%\\
\hline
\end{tabular}
\end{center}

\end{document}

