\documentclass[10dd, a5paper]{article}

\usepackage[utf8x]{inputenc}
\usepackage[russian]{babel}

\usepackage{xcolor}
\definecolor{kvant-paper}{RGB}{238,213,172}
\pagecolor{kvant-paper}

\usepackage{graphicx}
\graphicspath{{./}}

\usepackage{fontspec}
\setmainfont[
Extension = .ttf,
BoldFont = SchoolBook Bold Cyrillic,
ItalicFont = SchoolBook Italic Cyrillic]{SchoolBook Cyrillic}
\usepackage{multicol}

\usepackage[a5paper,lmargin=1.5cm,rmargin=1cm, tmargin=1.5cm, bmargin=1cm, footskip=0.2cm]{geometry}

\usepackage{sectsty}
\sectionfont{\fontsize{9dd}{8}\selectfont}

\usepackage{fancyhdr}
\usepackage{lipsum}
\newenvironment{Biglogo}
  {\par\medskip\noindent\minipage{\linewidth}}
  {\endminipage\par\vspace{1.5cm}}

\usepackage{tabularx}
\usepackage{layout}
\righthyphenmin=2
\begin{document}

\pagestyle{fancy}
\fancyhf{}
\renewcommand{\headrulewidth}{0pt}
\fancyfoot[RO]{\textbf{\LARGE\thepage}}
\noindent

\begin{minipage}[t][][t]{0.45\linewidth}
\begin{Biglogo}
\raggedright
\includegraphics[width=0.7\linewidth]{Kvant_logo}

\smallskip
\fontsize{9dd}{7dd}\selectfont
\raggedright
\textit{Основан в 1970 году}
\end{Biglogo}
\fontsize{8dd}{7dd}\selectfont
\raggedright
\itshape Научно-популярный физико-математический\\ журнал Академии наук СССР и Академии педагогических\\ наук СССР

\vspace{0.3cm}
\begin{minipage}[l]{0.1\linewidth}
\includegraphics[width=\linewidth]{nauka}
\end{minipage}%
\hspace{3dd}%
\begin{minipage}[с]{0.87\linewidth}
\fontsize{8dd}{6dd}\selectfont
\raggedright
Издательство <<Наука>>. Главная редакция физико-\\математической литературы
\end{minipage}
\end{minipage}%
\begin{minipage}[t][][b]{0.55\linewidth}

{\fontfamily{lmss}\fontsize{20dd}{6dd}\selectfont
\hspace{20dd}1988

{\fontsize{30dd}{6dd}\selectfont 4}
}
\vspace{1cm}

%Стилизуем содержаниее
\fontsize{17dd}{6dd}\selectfont{В номере:}

\fontsize{7dd}{6dd}\selectfont
\begin{tabularx}{\linewidth}{lX}
{\bf 2} & {\itshape И. Д. Новиков.} Вселенная как тепловая машина\\
{\bf 9} & {\itshape Ю. М. Львов, Л. А. Фейгн.} Ленгмюровские пленки --- путь к молекулярной электронике?\\
{\bf 14} & {\itshape М. Гарднер.} Рамсеевская теория графов\\
{\bf 21} & {\itshape} Интервью с американским математиком Рональдом Грэхемом\\
& \vspace{1dd}{\bf Задачник <<Кванта>>}\\
{\bf 27} & Задачи М1096-М1100, Ф110--Ф1112\\
{\bf 28} & Problems М109--М1100, Ф1108-Ф1112\\
{\bf 29} & Решения задач М1071, М1076-М1080, Ф1088-Ф1092\\
{\bf 36} & Избранные школьные задачи\\
& \vspace{1dd}{\bf <<Кванта>> для младших школьников }\\
{\bf 37} & Задачи\\
{\bf 38} & {\itshape А. П. Савин.} Десять цифры\\
& \vspace{1dd}{\bf Калейдоском <<Кванта>>}\\
{\bf 40} & Простые машины\\
& \vspace{1dd}{\bf Лаборатория <<Кванта>>}\\
{\bf 44} & {\itshape В. Ф. Яковлев} <<Физика для дураков>>\\
& \vspace{1dd}{\bf Математический кружок}\\
{\bf 48} & {\itshape Д. К. Фаддеев, М. С. Никулин, И. Ф. Соколовский.} Основной принцип дифференциального исчисления. Часть  {\MakeUppercase{\romannumeral 2}}: Свойства производной\\
& \vspace{1dd}{\bf Практикум абитуриента}\\
{\bf 44} & {\itshape В. Ф. Яковлев.} <<Физика для дураков>>\\
{\bf 54} & {\itshape В. А. Петров.} Неравенства и графики\\
{\bf 56} & {\itshape А. И. Буздин, С. С. Кротов.} Поверхностное натяжение и капиллярные явления\\
\vspace{1dd}&\\
{\bf 62} & {\bf Варианты вступительных экзаменов}\\
& \vspace{1dd}{\bf Информация}\\
{\bf 68} & Заочная физическая школа при МГУ\\
{\bf 69} & Омскому НОУ - 20 лет\\
{\bf 73} & {\bf Ответы, указания решения}\\
& {\bf <<Квант>> улыбается} {\it (61)}\\
& {\bf Смесь} {\it (72)}\\
& \vspace{1dd}{\bf Наша обложка}\\
{\bf 1} & {\it Читайте в этом номере статью на с. 2!}\\
{\bf 2} & {\it Этот рисунок Леонардо да Винчи, сделанный 500 лет назад, - своеобразный анонс <<Калейдоскопа <<Кванта>> из этого номера.}\\
{\bf 3} & {\it Шахматная страничка}\\
{\bf 4} & {\it Головоломка Рамсея.}\\
\end{tabularx}
\vspace{1cm}
\medskip
\raggedleft
\begin{minipage}[l]{0.05\linewidth}
{\fontsize{20dd}{6dd}\selectfont $\bullet$}
\end{minipage}%
\raggedright
\begin{minipage}[с]{0.95\linewidth}
\fontsize{6dd}{6dd}\selectfont
\raggedright
\textcopyright Издательство <<Наука>>.\\
Главная редакция физико математической литературы, <<Квант>>, 1988 
\end{minipage}
\end{minipage}
\end{document}
