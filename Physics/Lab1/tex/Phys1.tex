\documentclass[12pt, a4paper]{article}

\usepackage[utf8]{inputenc}
\usepackage[english, russian]{babel}
\usepackage{fancyhdr}
\usepackage{amsmath}
\usepackage{amsthm}
\usepackage{float}
\usepackage{graphicx}
\usepackage{tabularx}
\newcolumntype{L}{>{\raggedright\arraybackslash}X}
\usepackage{pgfplots}
\usepackage{float}
\usepackage{xcolor}
\usepackage{hyperref}
\usepackage{multirow}
\pgfplotsset{width=\textwidth, compat=1.13}

\usepgfplotslibrary{external}
\usepgfplotslibrary{fillbetween}
\usepgfplotslibrary{statistics}
\usetikzlibrary{patterns.meta}


\graphicspath{{./}}
\newcommand{\Mod}[1]{\ \mathrm{mod}\ #1}

\usepackage[a4paper, margin=1.5cm]{geometry}

\usepackage{titlesec}
\titlelabel{\thetitle.\quad}

\pagestyle{plain}

\fancypagestyle{firstpage}{%
  \chead{
  МИНИСТЕРСТВО НАУКИ И ВЫСШЕГО ОБРАЗОВАНИЯ РОССИЙСКОЙ ФЕДЕРАЦИИ 
ФЕДЕРАЛЬНОЕ ГОСУДАРСТВЕННОЕ АВТОНОМНОЕ  
ОБРАЗОВАТЕЛЬНОЕ УЧРЕЖДЕНИЕ ВЫСШЕГО ОБРАЗОВАНИЯ\bigskip

«Национальный исследовательский университет ИТМО»\bigskip

ФИЗИЧЕСКИЙ ФАКУЛЬТЕТ 
}
\fancyfoot[CO]{Санкт-Петербург, 2023}%
}



\definecolor{aqua}{HTML}{003844}
\definecolor{peri}{HTML}{5EB1BF}
\definecolor{royal_blue}{HTML}{0A2463}
\definecolor{periwinkle}{HTML}{D8DCFF}
\definecolor{cerulean}{HTML}{247BA0}
\definecolor{bloodred}{HTML}{690500}
\definecolor{imperial_red}{HTML}{FB3640}
\definecolor{purple}{HTML}{511730}
\definecolor{tangerine}{HTML}{FFA781}

\definecolor{blue1}{HTML}{54BEBE}
\definecolor{blue2}{HTML}{76C8C8}
\definecolor{blue3}{HTML}{98D1D1}
\definecolor{blue4}{HTML}{BADBDB}

\definecolor{pink1}{HTML}{C80064}
\definecolor{pink2}{HTML}{D7658B}
\newtheorem*{task}{Условие}
\newtheorem*{finish}{Заключение}

%\tikzexternalize
\begin{document}
\newgeometry{top=1.6cm,bottom=1.6cm, left = 1.2cm, right = 1.2cm}

\topskip0pt
\vspace*{0.25\textheight}
\begin{center}
\textbf{\LARGE РАБОЧИЙ ПРОТОКОЛ И ОТЧЁТ }

\LARGE по лабораторной работе №1.01

\LARGE <<Исследование распределения случайной величины>>

\end{center}
\vspace*{5cm}
\begin{flushright}
\begin{minipage}{.33\linewidth}
\textit{\textbf{Выполнил:}}\\
Хороших Дмитрий - P3217\\
\textit{\textbf{Преподаватель:}}\\
Коробков Максим Петрович
\end{minipage}
\end{flushright}


\thispagestyle{firstpage}
\newpage
\tableofcontents

\restoregeometry
\section{Введение}
\begin{enumerate}
\item Цель работы:

Вычисление распределения случайной величины на примере измерений амплитуд частот в записи шума и сравнение его с нормальным распределением.

\item Задачи:
	\begin{enumerate}
		\item[1.]  С помощью микрофона записать несколько секунд звукового шума.
		\item[2.] Получить массив амплитуд по всем записываемым частотам.
		\item[3.] Построить гистограмму распределения результатов измерений.
		\item[4.] Вычислить выборочное среднее и выборочное среднеквадратичное отклонение.
		\item[5.] Сравнить гистограмму с графиком функции нормального распределения (функции Гаусса) с таким же как и у экспериментального распределения средним значением и среднеквадратичным отклонением.
		\item[6.] Высчитать среднеквадратичное отклонение среднего значения и доверительный интервал для  этого среднего значения.
	\end{enumerate}
		
\item Объект исследования:

5-секундный звуковой файл с шумом (в формате .WAV).

\item Метод экспериментального исследования:

Однократный прямой замер амплитуд набора частот.

\item Рабочие формулы:

Функция Гаусса:
\begin{equation}
 \rho (t) = \frac{1}{\sigma \sqrt{2\pi}}\exp\left( -\frac{(t-\langle t \rangle)^2}{2 \sigma^2}\right)
\end{equation}

Выборочное среднее (приближение мат. ожидания):
\begin{equation}
\langle t \rangle_N = \frac{1}{N} \sum^N_{i=1} t_i
\end{equation}

Выборочное среднеквадратичное отклонение:

\begin{equation}
\sigma_N = \sqrt{\frac{1}{N-1}\sum^N_{i=1} (t_i - \langle t \rangle_N)^2}
\end{equation}

Среднеквадратичное отклонение среднего значения:
\begin{equation}
\sigma_{\langle t \rangle} = \sqrt{ \frac{1}{N * (N-1)} * \sum^N_{i=1} (t_i - \langle t \rangle_N)^2}
\end{equation}

Доверительный интервал для среднего значения амплитуды:
\begin{equation}
\Delta t = t_{\alpha, N} * \sigma_{\langle t \rangle} 
\end{equation}
\item Исходные данные:

Файл с полученными амплитудами доступен по ссылке: \url{https://github.com/Dimankarp}.

\item Измерительные приборы:

\begin{center}

\begin{table}[h!]
\begin{tabular}{|l|l|l|l|l|}
\hline
№ п/п & Наименование & Тип & Используемый диапазон & Погрешность прибора\\
\hline
1 & Микрофон WH-CH500 & Электронный & -  & -\\
\hline
\end{tabular}
\end{table}
\end{center}
\end{enumerate}
\section{Результаты прямых измерений и их обработка}
\subsection{Прямые измерения}

Получим список амплитуд (в единицах измерения dBFS) \footnote{dBFS (decibels to full scale) - опорный сигнал (мощность, напряжение) соответствует полной шкале аналого-цифрового преобразователя} из звукового файла и отразим его в таблице. В силу большого числа обрабатываемых данных в таблице приведены лишь начальные и конечные измерения. Полная таблица \ref{tab:1} доступна по ссылке: \url{https://github.com/Dimankarp}.

\begin{table}[h!]
\begin{center}
\begin{tabularx}{\linewidth}{l| l l L}
\hline
№ & $t_i$, dBFS & $t_i-\langle t \rangle_N$, dBFS & $(t_i-\langle t \rangle_N)^2$, $(dBFS)^2$\\
\hline
1 & 346.0 & 96.945 & 9398.294 \\ 
\hline 
2 & 333.0 & 83.945 & 7046.729 \\ 
\hline 
3 & 343.0 & 93.945 & 8825.625 \\ 
\hline
... & ... & ... & ... \\
\hline
4998 & 264.0 & 14.945 & 223.347 \\ 
\hline 
4999 & 288.0 & 38.945 & 1516.697 \\ 
\hline 
5000 & 323.0 & 73.945 & 5467.833 \\ 
\hline \\
 & $\langle t \rangle_N = 249.06 $ dBFS & $\sum^N_{t=1} (t_i - \langle t \rangle_N)=-7.63 * 10^{-11}$ dBFS &  $\sigma_N = 56.91$  dBFS  $\rho_{max}=0.007 $ $dBFS^{-1}$\\
 \hline
\end{tabularx}
\caption{Результаты прямых измерений}
\end{center}
\label{tab:1}
\end{table}

\subsection{Построение гистограммы и функции Гаусса}
Подготовим данные для построения гистограммы плотности распределения вероятности и графика функции нормального распределения. Аналогично, случаю с прямыми измерениями, таблица \ref{tab:2} приведена частично и доступна полностью по ссылке: \url{https://github.com/Dimankarp}.

\begin{table}[!h]
\begin{center}
\begin{tabular}{|l|l|l|l|l|}
\hline
Границы интервалов, $dBFS$ & $\Delta N$& $\frac{\Delta N}{N*\Delta t}$, $dBFS^{-1}$ & $t$, $dBFS$ & $\rho$, $dBFS^{-1}$\\
\hline 
[79.0;83.47]& 2 &  9e-05 & 81.235 & 9e-05 \\ 
\hline
[83.47;87.94] & 1 & 4e-05 & 85.705 & 0.00011 \\ 
\hline
[87.94;92.41] & 1 & 4e-05 & 90.175 & 0.00014 \\ 
\hline
... & ... & ... & ... & ...\\
\hline
[248.91;253.39] & 178 & 0.00796 & 251.15 & 0.00701 \\ 
\hline
[253.39;257.86] & 138 & 0.00617 & 255.625 & 0.00696 \\ 
\hline
[257.86;262.33] & 203 & 0.00908 & 260.095 & 0.00688 \\ 
\hline
... & ... & ... & ... & ...\\
\hline
[378.59;383.06] & 5 & 0.00022 & 380.825 & 0.00048 \\ 
\hline
[383.06;387.53] & 2 & 9e-05 & 385.295 & 0.0004 \\ 
\hline
[387.53;392.0] & 2 & 9e-05 & 389.765 & 0.00033 \\ 
\hline
\end{tabular}
\caption{Данные для построения гистограммы}
\end{center}
\label{tab:2}
\end{table}

Гистограмма плотности распределения вероятности и график функции нормального распределения приведены на рисунке \ref{gr:1}.

\begin{figure}[H]
\begin{tikzpicture}
\begin{axis}[
	axis lines = left,
	xlabel = \(t\),
	ylabel = {\(\rho\)},
	ymin=0,
	xmin=40,
	xmax=420,
	grid=both,
    grid style={line width=.1pt, draw=gray!10},
    major grid style={line width=.2pt,draw=gray!50},
    minor tick num=5,
	axis x line = bottom,
	axis line style ={line width = .3pt},
	ymax=0.01,
	legend style={at={(0.03,0.9)},anchor=west}
	]
\addplot[ybar, style = {fill=blue3, draw=blue1, mark=none, bar width = 5pt, fill opacity=0.9, draw opacity=0.9}] table [x, y, col sep=comma] {hist.csv};
\addplot[smooth, pink1, line width=2pt, dashed] table [x, y, col sep=comma] {normal.csv};
\legend{Гистограмма плотности вероятности, Функция нормального распределения}
\end{axis}
\end{tikzpicture}
\caption{Сравнение плотности распределения вероятности с функцией Гаусса (функцией нормального распределения)}
\label{gr:1}
\end{figure}

\subsection{Сравнение полученного распределения с нормальным}
При визуальном сравнение гистограммы и графика очевидно сходство их общей формы и некоторой симметрии относительно математического ожидания.  

Рассмотрим таблицу соответствия долей попавшихв $\sigma$-интервалы амплитуд к соответствующим значениям вероятности нормального распределения.
\begin{table}[!h]
\begin{center}
\begin{tabular}{|l|c|c|c|c|c|}
\hline
\multirow{2}{*}{} & \multicolumn{2}{c}{Интервал, $dBFS$} & \multirow{2}{*}{$\Delta N$} & \multirow{2}{*}{$\frac{\Delta N}{N}$} & \multirow{2}{*}{$P$}\\
\cline{2-3}
& от & до & & &\\
\hline 

$\langle t \rangle_N \pm \sigma_N$  & 192.146 & 305.964 &  3319& 0.664&$\approx 0.683$ \\
\hline
$\langle t \rangle_N \pm 2\sigma_N$  & 135.237 & 362.873 & 4781& 0.956&$\approx 0.954$\\
\hline
$\langle t \rangle_N \pm 3\sigma_N$ & 78.328 & 419.783 & 5000& 1.000 & $\approx 0.997$\\
\hline
\end{tabular}
\caption{Данные для построения гистограммы}
\end{center}
\label{tab:3}
\end{table}

Заметим, что значения $\frac{\Delta N}{N}$ доли попавших в интервал амплитуд близко к вероятности, соответствующей нормальному распределению. 

\subsection{Отклонение среднего значения и его доверительный интервал}
Вычислим среднеквадратичное отклонение среднего значения и его доверительный интервал.
\begin{equation*}
\sigma_{\langle t \rangle} = \sqrt{ \frac{1}{N * (N-1)} * \sum^N_{i=1} (t_i - \langle t \rangle_N)^2} \approx 0.805
\end{equation*}
Табличное значение коэффециент Стьюдента для числа измерений $N=5000$ и доверительной вероятности $\alpha=0.95$ равно $t_{\alpha, N}\approx 1.9604$.

Таким образом доверительный интервал равен:

\begin{equation*}
\Delta t = t_{\alpha, N} * \sigma_{\langle t \rangle} \approx 1.578 
\end{equation*}


\section{Вывод}
Таким образом, в ходе выполнения лабораторной работы удалось, взяв в качестве исходных данных массив амплитуд из аудиозаписи шума, вычислить распределение случайной величины и установить её схожесть с нормальным распределением.


\section{Приложение}
Проект этой лабораторной работы, содержащий полные таблицы, фалй с Python-кодом, использованным для вычисления, аудиозапись шума и исходны TeX-файлы доступен по ссылке: \url().

\end{document}



